% Other math stuff
% vector with prime symbol
\newcommand{\vecp}[1]{\ensuremath{\vec{#1}^{\,\prime}}}
\newcommand{\dvecp}[1]{\ensuremath{\dvec{#1}^{\,\prime}}}
\newcommand{\ddvecp}[1]{\ensuremath{\ddvec{#1}^{\,\prime}}}
% vector prime with subscript
\newcommand{\vecps}[2]{\ensuremath{\vec{#1}^{\,\prime}_{\!#2}}}

% copied from https://tex.stackexchange.com/a/44071/249769
% Macro \xvec
\makeatletter
\newlength\xvec@height%
\newlength\xvec@depth%
\newlength\xvec@width%
\newcommand{\xvec}[2][]{%
	\ifmmode%
	\settoheight{\xvec@height}{$#2$}%
	\settodepth{\xvec@depth}{$#2$}%
	\settowidth{\xvec@width}{$#2$}%
	\else%
	\settoheight{\xvec@height}{#2}%
	\settodepth{\xvec@depth}{#2}%
	\settowidth{\xvec@width}{#2}%
	\fi%
	\def\xvec@arg{#1}%
	\def\xvec@dd{:}%
	\def\xvec@d{.}%
	\raisebox{.2ex}{\raisebox{\xvec@height}{\rlap{%
				\kern.05em%  (Because left edge of drawing is at .05em)
				\begin{tikzpicture}[scale=1]
					\pgfsetroundcap
					\draw (.05em,0)--(\xvec@width-.05em,0);
					\draw (\xvec@width-.05em,0)--(\xvec@width-.15em, .075em);
					\draw (\xvec@width-.05em,0)--(\xvec@width-.15em,-.075em);
					\ifx\xvec@arg\xvec@d%
					\fill(\xvec@width*.45,.5ex) circle (.5pt);%
					\else\ifx\xvec@arg\xvec@dd%
					\fill(\xvec@width*.30,.5ex) circle (.5pt);%
					\fill(\xvec@width*.65,.5ex) circle (.5pt);%
					\fi\fi%
				\end{tikzpicture}%
	}}}%
	#2%
}
\makeatother

% --- Override \vec with an invocation of \xvec.
\let\stdvec\vec
\let\ovec\vec
\renewcommand{\vec}[1]{\xvec[]{#1}}
% --- Define \dvec and \ddvec for dotted and double-dotted vectors.
\newcommand{\dvec}[1]{\xvec[.]{#1}}
\newcommand{\ddvec}[1]{\xvec[:]{#1}}
