\documentclass[8pt]{beamer}
\usetheme{Copenhagen}
\usepackage[fleqn]{mathtools}
\usepackage{stmaryrd}
\usepackage[numbers]{natbib}

\title{Lagrange Polynomials}
\date{16 July 2024}
\author{Simon Binder, Mihail Prudnikov}
\begin{document}
	\begin{frame}[plain]
		\maketitle
	\end{frame}
	
	\begin{frame}{Definitions 1}
		\begin{block}{Basis}
			Let $V$ be a vector space over a field $K$ $B :=\{ v_1, \ldots, v_n\}\subset V $ is a basis of $V$ if and only if:
			\begin{itemize}
				\item vectors in B are linearly independent
				\item
				 $\forall v \in V: \exists\alpha_i \in K:\forall i\in 1,\ldots,n:  v = \sum_{i=1}^{n}\alpha_i v_i$ 
			\end{itemize}\cite{basis}
		\end{block}
		\begin{block}{Lagrange Polynomials}
			Let $\{x_0, \ldots, x_n \} \subset \mathbb{R}: \forall i,j \in 0, \ldots ,n: x_i \neq x_j \text{ if } i \neq j$. For this set of values, we can define Lagrange polynomials: $\{\ell_0(x), \ldots, \ell_n(x)\}$ with \\ $\ell_j(x) = 
			\displaystyle \prod_{\substack{i= 0 \\ i \neq j }}^{n}\frac{x - x_i}{x_j - x_i}$. \cite{lagrange}
		\end{block}
	\end{frame}
	
	\begin{frame}{Definitions 2}
		\begin{block}{Dimension}
			The cardinality of a basis. The maximal cardinality of a set of vectors that can be linearly independent. \cite{basis}
		\end{block}
		\begin{block}{$P_n$}
			$P_n$ is a vector space of polynomials defined as $P_n = \{p \in R[X] : \deg(p) \leq n \}$ with $n \in \mathbb{N}_0$ where $R[X]$ is the ring of polynomials.\cite{lagrange}
		\end{block}
	\end{frame}
	
	\begin{frame}{Proofs 1}
		\begin{block}{Theorem 1}
			$\ell_j(x_i) = \delta_{ij}$
		\end{block}
		\begin{proof}
			Let $x_i, x_j \in \{x_0, \ldots, x_n \}$ and $j \neq i$ then we conclude: 
			\begin{align*}
				\ell_j(x_i) &= \prod_{\substack{k= 0 \\ k \neq j }}^{n}\frac{x_i - x_k}{x_j - x_k} = \prod_{\substack{k= 0 \\ k \neq j}}^{i-1}\frac{x_i - x_k}{x_j - x_k} \cdot 0 \cdot \prod_{\substack{k= i+1 \\ k \neq j}}^{n}\frac{x_i - x_k}{x_j - x_k} = 0 \\
				\ell_j(x_j) &= \prod_{\substack{k= 0 \\ k \neq j }}^{n}\frac{x_j - x_k}{x_j - x_k} =  \prod_{\substack{k= 0 \\ k \neq j }}^{n} 1 = 1
			\end{align*}
		\end{proof}
	\end{frame}
			
	\begin{frame}{Proofs 2}
		\begin{block}{Theorem 2}
			Monomials of degree $0$ to $n$ form a basis of $P_n$. with $n \in \mathbb{N}_0: n\leq \infty$.
		\end{block}
		\begin{proof}
 			we can evalueate the polynomial at $x = 0$ for the polynomial and all of its derivitives
			\begin{align*}
				0 &= \sum_{i=0}^{n}\alpha_i x^i \Rightarrow 0  =\sum_{i=1}^{n}\alpha_i \cdot 0 + \alpha_0 \Rightarrow \alpha_0 = 0\\
				\Rightarrow \frac{d}{dx} 0 = 0 &=\sum_{i=1}^{n}i\alpha_i x^{i-1} \Rightarrow 0  =\sum_{i=2}^{n}i\alpha_i \cdot 0 + \alpha_1 \Rightarrow \alpha_1 = 0\\
			\end{align*}
			repeat the argument till all of $\alpha_i$ are evaluated $\Rightarrow \forall i \in 0, \ldots, n: \alpha_i = 0\Rightarrow$ the polynomials are linearly independant\\
		\end{proof}

		\end{frame}
			
		\begin{frame}{Proofs 3}
			\begin{proof}
				Using monomials, we can also construct any polynomial up to the degree of $n$ because
				\begin{align*}
					\forall p \in P_n : p = \sum_{i=1}^{n}\alpha_i x^i 
				\end{align*}
				because of how polynomials are defined. Hence, the monomials form a basis of $P_n$\\
			\end{proof}
			Because we require $n+1$ monomials to construct any polynomial in $P_n$, we can say that $\dim(P_n) = n+1$.
			
		\end{frame}
							
		\begin{frame}{Proofs 4}
			\begin{theorem}
				For $\{x_0, \ldots x_n \}\subset \mathbb{R}$, the corresponding Lagrange polynomials $\{\ell_0, \ldots \ell_n \}$ will be linearly independent in $P_n$.
			\end{theorem}
			\begin{proof}
				\begin{itemize}
					\item 
					Proof that $\forall j \in 0\ldots n: \ell_j \in P_n$\\
					\begin{align*}
						\displaystyle deg(\ell_j) =& deg\bigg(\prod_{\substack{k= 0 \\ k \neq j }}^{n}\frac{x - x_k}{x_j - x_k}\bigg) =deg\bigg( \prod_{\substack{k= 0 \\ k \neq j }}^{n}(x-x_k)\bigg) \\=& \sum_{\substack{k= 0 \\ k \neq j }}^{n} deg(x-x_k) = \sum_{\substack{k= 0 \\ k \neq j }}^{n} 1 = n\\
						\Rightarrow& \ell_j \in P_n
					\end{align*}
				
				\end{itemize}
			\end{proof}
		\end{frame}
		\begin{frame}{Proofs 5}
			\begin{proof}
				\begin{itemize}
					\item Proof of linear independence:\\
					Let's assume $\{\ell_0, \ldots \ell_n \}$ is linearly dependent and use the $\delta_{ij}$ property of $\ell_j$
					\begin{align*}
						&\exists j \in \{1, \ldots ,n \} : \ell_j(x) = \sum_{\substack{i= 0 \\ i \neq j }}^{n} \alpha_i \ell_i(x)
						\Rightarrow  \ell_j(x_j) = \sum_{\substack{i= 0 \\ i \neq j }}^{n} \alpha_i \ell_i(x_j)\\
						&\Rightarrow 1 = \ell_j(x_j) = \sum_{\substack{i= 0 \\ i \neq j }}^{n} \alpha_i \underbrace{\ell_i(x_j)}_{=0} = 0 \lightning\\& \Rightarrow \text{Lagrange polynomials are linearly independent}
					\end{align*}
				\end{itemize}
			\end{proof}
		\end{frame}
									
		\begin{frame}{Proofs 6}
			\begin{theorem}
				Let $V$ be a vector space, $M = \{v_1, \ldots, v_n\}$, $\# M = \dim(V)$, $M \subset V$ and $\{v_1, \ldots, v_n\}$ are linearly independent, then $M$ is a basis of $V$.
			\end{theorem}
			\begin{proof}
				Let there be one more vector in $V$ that is linearly independent of vectors in $M$, then $\dim(V) \geq n+1$ due to how the dimension is defined, which is a contradiction.\\ 
				We now know that $\forall u \in V$ $\{v_1, \ldots, v_n, u\}$ will be linearly dependent.
				\begin{align*}
					&\Rightarrow \forall u \in V: \forall i \in 1, \ldots, n: \exists \alpha_i \in K : u = \sum_{i = 1}^{n}\alpha_i v_i 
				\end{align*}
				And, because vectors in $M$ are linearly independent, $M$ must be the basis of $V$, because of how the basis is defined.
			\end{proof}
		\end{frame}
											
		\begin{frame}{Proofs 7}
			\begin{theorem}
				Lagrange polynomials $\{\ell_0,\ldots, \ell_n\}$ are a basis of $P_n$ for $n \in \mathbb{N}_0$.
			\end{theorem}
			\begin{proof}
				We know that $\#\{\ell_0,\ldots, \ell_n\} = n+1 = \dim(P_n)$. The Lagrange polynomials are linearly independent and $\{\ell_0,\ldots, \ell_n\} \subset P_n$. according to our previous theorem, $\{\ell_0,\ldots, \ell_n\}$ is a basis of $P_n$.
			\end{proof}
		\end{frame}
		\begin{frame}{Sources}
			\begin{thebibliography}{9}
				\setbeamertemplate{bibliography item}[number]
				\bibitem{basis}
				Brown William A. (1991), Matricies and vector spaces, New York M. Dekker, ISBN 978-0-8247-8419-5 page 107
				\bibitem{lagrange}
				Rannacher, R. 2017. Numerik 0: Einführung in die Numerische Mathematik. Heidelberg University Publishing. Page 24
			\end{thebibliography}
		\end{frame}
													
\end{document}
