\documentclass[8pt]{beamer}
\usetheme{Copenhagen}



\title{Lagrange Polynomials}
\date{16 July 2024}
\author{Simon Binder, Mihail Prudnikov}
\begin{document}
\begin{frame}[plain]
    \maketitle
\end{frame}
\begin{frame}{Definitions 1}
	\begin{block}{Basis}
		let $V$ be a vectorspace above a field $K$ and $v_1 \ldots v_n \in V$ $\{ v_1 \ldots v_n\} $ are a Basis of $V$ iff $\forall v \in V: v = \sum_{i =1}^{n}\alpha v_i$ and $\{ v_1 \ldots v_n\}$ are linearly indipendant. With $\forall i \in 1\ldots n: \alpha_i \in K $
	\end{block}
	\begin{block}{lagrange poylnomials}
		let $\{x_0\ldots x_n \}$ be a set of values with $x_i \neq x_j$ if $i \neq j$. For this set of values we can define lagrange polynomials: $\{\ell_0(x)\ldots \ell_n(x)\}$ with $\ell_j(x) = 
		\displaystyle \prod_{\substack{i= 0 \\ i \neq j }}^{n}\frac{x - x_i}{x_j - x_i}$ 
	\end{block}
\end{frame}

\begin{frame}{Definitions 2}
	\begin{block}{Dimension}
		the cardinality of a basis. The maximal cardinality of a set, of vectors, that can be linearly indipendant. 
	\end{block}
	\begin{block}{$P_n$}
		$P_n$ is a vectorspace of polynomials defined as $P_n = \{p \in R[X]: deg(p)\leq n \}$ with $n \in \mathbb{N}_0$ where $ R[X]$ is the ring of polynomials
	\end{block}
\end{frame}
\begin{frame}{Proofs 1}
	\begin{block}{theorem 1}
		$\ell_j(x_i) = \delta_{ij}$
	\end{block}
	\begin{proof}
		let $x_i \in \{x_0\ldots x_n \}$ and $j \neq i $ then we conclde: 
		\begin{align*}
			\ell_j(x_i) &= \prod_{\substack{k= 0 \\ k \neq j }}^{n}\frac{x_i - x_k}{x_j - x_k} = \prod_{\substack{k= 0 \\ k \neq j}}^{i-1}\frac{x_i - x_k}{x_j - x_k} \cdot 0 \cdot \prod_{\substack{k= i+1 \\k \neq j}}^{n}\frac{x_i - x_k}{x_j - x_k} = 0 \\
			\ell_j(x_j) &= \prod_{\substack{k= 0 \\ k \neq j }}^{n}\frac{x_j - x_k}{x_j - x_k} =  \prod_{\substack{k= 0 \\ k \neq j }}^{n} 1 = 1
		\end{align*}
	\end{proof}
\end{frame}
\begin{frame}{Proofs 2}
\begin{block}{theorem 2}
	monoms from degree $0$ to $n$ make up the basis of the vectorspace of Polynomials of degree $x \leq n$, also called $P_n$, with $n \in \mathbb{N}_0$
\end{block}
	\begin{proof}
	Because monoms have different degrees, and hence can't cancel eachother out. Therefore: 
	\begin{align*}
		0 &= \sum_{i =0}^{n}\alpha_i x^i \Leftrightarrow \alpha_i = 0\text{   } i\in 0\ldots n \\
	\end{align*}
	using monoms we can also construct any polynomial up to the degree of n becuase
	\begin{align*}
		\forall p \in P_n : p = \sum_{i =0}^{n}\alpha_i x^i 
	\end{align*}
	because of how polynomials are definied. Hence the monoms form a basis of $P_n$
	\end{proof}
	because we require $n+1$ monoms we can say, that $dim(P_n) = n+1$
\end{frame}

\begin{frame}{Proofs 3}
	\begin{theorem}
		for the set of values $\{x_0, \ldots x_n \}$ the according lagrange polynomials $\{\ell_0, \ldots \ell_n \}$ will be linearly indipendant in $P_n$
	\end{theorem}
	 \begin{proof}
		\begin{itemize}
			\item 
				proof that $\ell_j \in P_n$:\\
				$\displaystyle \ell_j = \prod_{\substack{k= 0 \\ k \neq j }}^{n}\frac{x - x_k}{x_j - x_k}$ hence the polynomial is defined by some constant $\displaystyle \prod_{\substack{k= 0 \\ k \neq j }}^{n}\frac{1}{x_j - x_k}$ and $ n$ linear factors $(x-x_k)$ each multiplication of a linear factor with another raises the degree by one $\Rightarrow deg(\ell_j(x)) = n \Rightarrow \ell_j \in P_n$
			 \item proof linear indipendance:\\
			 lets assume $\{\ell_0, \ldots \ell_n \}$ are linearly dipendant. 
			 \begin{align*}
			 	&\exists j \in \{1, \ldots ,n \} \forall i \in \{1, \ldots ,n \}: i \neq j: \ell_j(x) = \sum_{\substack{k= 0 \\ k \neq j }}^{n} \alpha_i\ell_i(x)
			 \end{align*}
		\end{itemize}
	 \end{proof}
\end{frame}
\end{document}
