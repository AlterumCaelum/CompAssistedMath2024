\documentclass{beamer}
\usepackage[T1]{fontenc}
\usepackage[utf8]{inputenc}
\usepackage{listings}
\usepackage{amssymb}


\def\lstlanguagefiles{lstlean.tex}
\lstset{language=lean}

\usepackage{fontspec}
% switch to a monospace font supporting more Unicode characters
\setmonofont{FreeMono}
\usepackage{minted}
\newmintinline[lean]{lean4}{bgcolor=white}
\newminted[leancode]{lean4}{fontsize=\footnotesize}
\usemintedstyle{tango}  % a nice, colorful theme

% Theme
\usetheme{Madrid}

% Title page
\title{Convergence of the nth Root of n}
\author{Your Name}
\date{\today}

\begin{document}

\begin{frame}
    \titlepage
\end{frame}

\begin{frame}[fragile]{Definitions}
    % Your content for the definitions section goes here
    \begin{definition}
        A sequence $(a_n)$ is said to converge to a limit $x$ if for every $\epsilon > 0$, there exists an $N \in \mathbb{N}$ such that for all $n \geq N$, $|a_n - x| < \epsilon$.
        \begin {align*}
            \lim_{n \to \infty} a_n = x \iff \forall \epsilon > 0, \exists N \in \mathbb{N} \text{ such that } \forall n \geq N, |a_n - x| < \epsilon
        \end {align*}
    \end{definition}
    
    \begin{block}{Lean Code}
    \begin{leancode}
        def ConvergesTo (a : ℕ → ℝ) (x : ℝ) : Prop := 
            ∀ ε > 0, ∃ (n : ℕ), ∀ m ≥ n, |a m - x| < ε
    \end{leancode}
    \end{block}

\end{frame}

\begin{frame}[fragile]{Lemmas}
    % Your content for the lemmas section goes here
    \begin{lemma}
        Let $(a_n)$ be a convergent sequence with limit $L$. Then, for any constant $c$, the sequence $(c \cdot a_n)$ is also convergent with limit $c \cdot L$.
    \end{lemma}
\end{frame}

\begin{frame}{Math Proof}
    % Your content for the math proof section goes here
\end{frame}

\begin{frame}{Proof in Lean}
    % Your content for the proof in Lean section goes here
\end{frame}

\end{document}