\documentclass{beamer}
\usepackage[T1]{fontenc}
\usepackage[utf8]{inputenc}
\usepackage{listings}
\usepackage{amssymb}


\def\lstlanguagefiles{lstlean.tex}
\lstset{language=lean}

\usepackage{color}
\definecolor{keywordcolor}{rgb}{0.7, 0.1, 0.1}   % red
\definecolor{tacticcolor}{rgb}{0.0, 0.1, 0.6}    % blue
\definecolor{commentcolor}{rgb}{0.4, 0.4, 0.4}   % grey
\definecolor{symbolcolor}{rgb}{0.0, 0.1, 0.6}    % blue
\definecolor{sortcolor}{rgb}{0.1, 0.5, 0.1}      % green
\definecolor{attributecolor}{rgb}{0.7, 0.1, 0.1} % red

% Theme
\usetheme{Madrid}

% Title page
\title{Convergence of the nth Root of n}
\author{Your Name}
\date{\today}

\begin{document}

\begin{frame}
    \titlepage
\end{frame}

\begin{frame}[fragile]{Definitions}
    % Your content for the definitions section goes here
    \begin{definition}[Convergence of a Sequence]
    A sequence $(a_n)$ is said to converge to a limit $L$ if for every positive real number $\epsilon$, there exists a positive integer $N$ such that for all $n \geq N$, $|a_n - L| < \epsilon$.
    \end{definition}

    \begin{lstlisting}
    import data.real.basic

    def is_convergent (a : ℕ → ℝ) (L : ℝ) : Prop :=
    ∀ ε > 0, ∃ N, ∀ n ≥ N, abs (a n - L) < ε
    \end{lstlisting}

\end{frame}

\begin{frame}[fragile]{Lemmas}
    % Your content for the lemmas section goes here
    \begin{lemma}
        For any positive integer $n$, the $n$th root of $n$ to the power of $n$ is $n$ i.e. $\sqrt[n]{n}^n = n$.
    \end{lemma}

    \begin{lemma}
    \begin{lstlisting}
    lemma nthRoot_pow (n : ℕ) (h : n ≥ 1) : (n.root n) ^ n = n
    \end{lstlisting}
    \end{lemma}
\end{frame}

\begin{frame}{Math Proof}
    % Your content for the math proof section goes here
\end{frame}

\begin{frame}{Proof in Lean}
    % Your content for the proof in Lean section goes here
\end{frame}

\end{document}