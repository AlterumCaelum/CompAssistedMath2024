\documentclass{beamer}
\usepackage[T1]{fontenc}
\usepackage[utf8]{inputenc}
\usepackage{listings}
\usepackage{amssymb}

\usepackage{fontspec}
% switch to a monospace font supporting more Unicode characters
\setmonofont{FreeMono}
\usepackage{minted}
\newmintinline[lean]{lean4}{bgcolor=white}
\newminted[leancode]{lean4}{fontsize=\footnotesize}
\usemintedstyle{tango}  % a nice, colorful theme

% Theme
\usetheme{Madrid}

% Title page
\title{Convergence of the nth Root of n}
\author{Your Name}
\date{\today}

\begin{document}

\begin{frame}
    \titlepage
\end{frame}

\begin{frame}[fragile]{Definitions}
    % Your content for the definitions section goes here
    \begin{definition}
        A sequence $(a_n)$ is said to be bounded if there exists a real number $C$ such that $|a_n| \leq C$ for all $n \in \mathbb{N}$.
    \end{definition}
    \begin{block}{Lean Code}
        \begin{leancode}
        def IsBounded (a : ℕ → ℝ) : Prop :=
        ∃ (C : ℝ), ∀ (n : ℕ), |a n| ≤ C
        \end{leancode}
    \end{block}

\end{frame}

\begin{frame}[fragile]{Lemmas}
    % Your content for the lemmas section goes here
    \begin{lemma}
        For any positive integer $n$, the $n$-th root of $n$ raised to the power of $n$ is equal to $n$, i.e. $(\sqrt[n]{n})^n = n$.
    \end{lemma}

    \begin{block}{Lean Code}
    \begin{leancode}
    lemma nthRoot_pow (n : ℕ) (h : n ≥ 1) : (n.root n) ^ n = n
    \end{leancode}
    \end{block}
\end{frame}

\begin{frame}{Math Proof}
    % Your content for the math proof section goes here
    Claim: The sequence of $n$-th roots of $n$ converges to 1.
    \begin{proof}
        Let $(a_n)$ be the sequence defined by $a_n = \sqrt[n]{n}$.

        First, note that for any positive integer $n \geq 1$, we have $1 \leq \sqrt[n]{n}$. This is because $1^n = 1 \leq n = \sqrt[n]{n}^n$, and taking the $n$-th root preserves the inequality. Therefore applies $\sqrt[n]{n} = 1 + a_n$ for a suitbable sequence $(a_n)$.

        Second, we have $ \sqrt[n]{n} \leq 1 + 2 / \sqrt[2]{n}$. This follows for $n \geq 2$ because $n = \sqrt[n]{n}^n = (1 + a_n)^n = \sum_{k=0}^{n} \binom{n}{k} * a_n ^k \geq \binom{n}{2} * a_n^2 = \frac{n!}{(n-2)! * 2!} * a_n^2 =\frac{n * (n-1)}{2} * a_n^2$ . 
        We can conclude now $n \geq \frac{n * (n-1)}{2} * a_n^2
        \Leftrightarrow a_n^2 \leq \frac{2}{n-1} \Leftrightarrow a_n \leq \sqrt[2]{\frac{2}{n-1}} \leq \frac{2}{\sqrt[2]{n-1}}$

       We also know:
        \[
        \lim_{n \to \infty} 1 = 1
        \]
        \[
        \lim_{n \to \infty} (1 + 2 / \sqrt[2]{n}) = 1
        \]

        Since $1 \leq \sqrt[n]{n} \leq 1 + 2 / \sqrt[2]{n}$ for all $n \in \mathbb{N}$, by the sandwich theorem, we conclude that $\lim_{n \to \infty} \sqrt[n]{n} = 1$.

        Therefore, the sequence of $n$-th roots of $n$ converges to 1.
    \end{proof}

\end{frame} 

\begin{frame}{Proof in Lean}
    % Your content for the proof in Lean section goes here
\end{frame}

\end{document}